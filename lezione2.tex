\documentclass[a5paper]{article}

% Load packages
\usepackage[latin1]{inputenc}
\usepackage[T1]{fontenc}
\usepackage[italian,english]{babel}
\usepackage{listings}

% Set the title
\author{Andrea Brancaleoni}
\title{Lezione2}
\date{2012/03/06 12:35:29}

% Start Document
\begin{document}
  \maketitle

  \section{Programmazione ad oggetti}
    Pi� il linguaggio � tipizzato pi� ci aiuta ad evitare problemi. In un linguaggio fortemente tipizzato

    \begin{lstlisting}
      Data d;
      int g;
      d = g; // Illegale.
    \end{lstlisting}

    errori di questo tipo sono rilevati a compile time e non a runtime. Il controllo dei tipo, secondo il professore, � un vantaggio. Ci sono molti linguaggi che sono molto poco tipizzati, ad esempio ruby o python.

    \subsection{Le modifiche sono costose}
      \begin{itemize}
        \item una modifica ad una struttura comporta tante altre piccole modifiche a meno di mascherare le strutture.
        \item fare tante piccole modifiche �:
          \begin{itemize}
            \item fonte di errori.
            \item lungo e costoso.
            \item difficile se la documentazione � cattiva.
          \end{itemize}
        \item la realizzazione interna � una delle parti pi� soggette a cambiamenti, una volta che ho una interfaccia la struttura interna la posso nascondere in modo da poterla modificare successivamente senza dover modificare ulteriormente il codice dipendente dall'interfaccia precedentemente definita.
      \end{itemize}

      Con un linguaggio ad oggetti se non rispettiamo le interffacce e i tipi, diversamente da uno che non lo �, non compiliamo.

    \subsection{ADT}
      Astrazione sui dati che non li classifica in base alla lora rappresentazione. Il comportamento dei dati � espesso in termini di operazioni.
      
      Un ADT,
      Esporta:
        \begin{itemize}
          \item il nome del tipo
          \item l'interfaccia
        \end{itemize}
      Nasconde:
        \begin{itemize}
          \item la struttura del tipo
          \item l'implementazione delle operazioni
        \end{itemize}

\end{document}
