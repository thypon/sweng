\documentclass[a5paper]{}

% Load packages
\usepackage[latin1]{inputenc}
\usepackage[T1]{fontenc}
\usepackage[italian,english]{babel}

% Set the title
\author{Andrea Brancaleoni}
\title{Lezione5}
\date{2012/03/13 12:34:01}

% Start Document
\begin{document}
  \maketitle

  \section{Overriding}
    una sottoclasse pu� ridefinire l'implementazione di un metodo, tuttavia:
    \begin{itemize}
      \item L'intestazione del metodo non deve cambiare
      \item \sout{L'intestazine non include il tipo retituito, che quindi pu� cambiare con la regola della covarianza.} Non la vedremo durante il corso quindi useremo solo la prima delle due regole.
    \end{itemize}

    \begin{lstlisting}
      public class AutomobileElettrica extends Automobile {
        private boolean batterieCariche;
        public voi ricarica() { batterieCariche = true; }
          . . .

        @Override
        public void accendi() {
        }
      }
    \end{lstlisting}

    Esiste anche la possibilit� di ridefinire un metodo con lo stesso nome del metodo ereditario ma con intestazione diversa non avremmo un caso di overriding ma piuttosto un caso di overloading.

  \section{Super}
    In java abbiamo anche un'altra variabile speciale oltre a this. Il suo nome � super ed � il modo per accedere da un sottometodo della sottoclasse ai metodi della superclasse.

    E' possibile anche usarlo all'interno del costruttore richiamando super (senza riferimento a metodi della classe) applicato negli argomenti desiderati da uno dei costruttori pubblici del padre.

  \section{Object}
    Ogni classe di cui non definiamo esplicitamente la classe Ancestor eredit� dall'oggetto java.lang.Ojbect.

    La classe Object definisce i seguenti metodi:
    \begin{itemize}
      \item public boolean equals(Object);
      \item public String toString();
      \item public Object clone();
    \end{itemize}

  \section{String}
    L'implementazione in Java delle stringhe, ha una serie di metodi predefiniti ed ogni sua istanza � immutabile.

    \begin{lstlisting}
      public class ProvaString {
        String a = new String(); // a e' un riferimento alla stringa vuota
        String b = new String(``Ciao''); // b e' il riferimento alla stringa 'ciao'
        String c = new String(b); // b e' una nuova stringa il cui contenuto e' uguale a quello di b
        String d = b; // d e' la stringa b
        System.out.println(b + `` '' + c + `` '' + d);
      }
    \end{lstlisting}

  \section{Visibilit�}

    \subsection{Package}
      Compilation unit ha come regola di visibilit� il package, e serve ad avere file con nomi diverse che stanno nella stessa directory.

      Noi possiamo distinguere all'interno del package classi che sono pubbliche nel package o private nel package, quelle che sono pubbliche nel package vengono chiamate friendly e il loro identifier non � mostrato.
    \subsection{Compilation Unit}
      Un file contiene la dichiarazione di una o pi� classi/interfacce, ma una sola di queste classi pu� essere dichiarata pubblica e deve avere lo stesso nome del file.
      C'� al pi� un metodo main. (Il professore sconsiglia di avere un main per ogni classe, il TIJ fa il contrario).
    \subsection{Visibility}
      public:
      \begin{itemize}
        \item Visibili a tutti attraverso l'input del package
        \item Il file deve avere lo stesso nome della classe pubblica
        \item al pi� una public class per ogni compilation unit/file
      \end{itemize}
      
      ``friendly'':
      \begin{itemize}
        \item sono visibili solo all'interno dello stesso package/compilation/unit
        \item possono stare in un file con altre classi
      \end{itemize}

      private:
      \begin{itemize}
        \item Sono visibili solo all'interno dello stesso scope
      \end{itemize}
    
    \subsection{Attributi e metodi}
      Attributi e metodi di una classe vengono sempre ereditati e possono essere:
      \begin{itemize}
        \item public, sono visibili a tutti
        \item private, sono visibili all'interno della classe, non sono visibili nelle sottoclassi
        \item protected, sono visibili alle classi nello stesso package, sono visibili anche nelle sottoclassi
        \item ``friendly'', sono visibili alle classi nello stesso package, sono visibili solo alle sottoclassi nello stesso package. ``friendly'' non � una parola di java, venne usata per la prima volta da Bruce Eckel all'interno di TIJ.
      \end{itemize}

  \section{Polimorfismo}
    Una classe definisce un tipo. Una sottoclasse definisce un sottotipo. Un oggetto del sottotipo � sostitiubile a un oggeto del tipo superiore.

    Si distingue tra:
      \begin{itemize}
        \item tipo statico. Il tipo dichiarato
        \item tipo dinamico. il tipo dell'oggetto attualmente assegnato.
      \end{itemize}

      Esempio:
      \begin{lstlisting}
        public class UsaAutomobile {
          public static void partenza(Automobile p) {
            if (p.puoPartire())
              p.accendi()
          }

          public static void main(String[] args) {
            Automobile myCar = new AutomobileElettrica(``T'');
            partenza(myCar);
          }
        }

      \end{lstlisting}

      In java una variabile di qualsiasi tipo T pu� riferirsi a un qualunque oggetto di tipo T o a una sua qualunque sottoclasse.

    \subsection{Tipo statico e tipo dinamico}
      Il tipo statico � quello definito in dichiarazione. Il metodo dinamico si cambia con il costruttore.

      Il compilatore in generale non definisce il metodo ma cerca dinamicamente il codice in base a come � defito dinamicamente (ora mi verrebbe da chiedere cosa sarebbe costato aggiungere le lambda).
      
  \section{Astratti}
    Un metodo � astratto quando non viene specificata l'implementazione.

    Una classe � astratta se viene definito almeno un metodo astratto.

    Non � possibile creare istanza di una classe astratta. E' utile per definire astrazioni di alto livello.

    \begin{lstlisting}
      abstract class Shape {
        static Screen screen = new Screen();
        Shape() {}
        abstract void show();
      }

      class Circle extends Shape {
        void show() {
        }
      }
    \end{lstlisting}

    Shape s = new Shape(); // Errato
    Circle c = new Circle(); // Corretto
    Shape s = new Circle(); // Corretto in quanto abbiamo l'interfaccia

    \section{Final}
      Final permette di bloccare un metodo/variabile o un'intera classe da essere modificata nei figli successivi.
\end{document}
