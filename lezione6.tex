\documentclass[a5paper]{article}

% Load packages
\usepackage[latin1]{inputenc}
\usepackage[T1]{fontenc}
\usepackage[italian,english]{babel}

% Set the title
\author{Andrea Brancaleoni}
\title{Lezione6}
\date{2012/03/14 10:28:30}

% Start Document
\begin{document}
  \maketitle

  \section{Ereditariet�}
    \subsection{Limiti dell'ereditariet� semplice}
      \begin{itemize}
        \item L'ereditariet� semplice non permette la descrizione di numerose situazioni reali. Ad esempio supponiamo di avere una classe Giocattoo ed una classe Automobili. In assenza di ereditariet� multipla non posso definire la classe AutomobileGiocattolo
        \item La soluzione java distingue tra una gerarchia di ereditariet� (semplice) ed una gerarchia di implementazione (multipla) introducendo il costrutto interface. Non avendo l'ereditariet� multipla sulle classi abbiamo l'ereditariet� multipla su un nuovo concetto detto interfaccia.
      \end{itemize}

    \subsection{Interfacce}
      Dovrebbero essere la sublimazione dell'information hiding. Nascondiamo a un probabile utilizzatore. E' un oggetto i cui metodi sono tutti astratti, gli attributi sono static e finale per definizione. Questo vuol dire che io sto semplicemente dichiarando attraverso un'interfaccia le parti che voglio condividere.

      Una volta che ho definito l'interfaccia dovr� definire anche l'implementazione di questa interfaccia.

      Serve a disaccoppiare cosa avviene da come implementiamo poi il nostro codice. Le classi astratte servono a gestire Late binding, polimorfismo ed ereditariet�.

      \begin{lstlisting}
        interface <nome> extends <nome1>, <nome2> \ldots {
          \ldots
        }
      \end{lstlisting}

      \begin{lstlisting}
        public interface Shape {
          public String baseclass = ``shape'';
          public void draw( );
        }

        public class Circle implements Shape {
          public void draw() {
            System.out.println(``Drawing Circle here'');
          }
        }

        public class Main {
          public static void main(String[] args) {
            Shape circleshape = new Circle(); // Una interfaccia come le classi astratte non possono essere istanziate. L'unica cosa che posso fare � istanziare una classe che la implementa.
            circleshape.draw(); // D'ora in poi posso usare solo i metodi definiti nell'interfaccia.
          }
        }
      \end{lstlisting}

    \subsection{Gerarchia delle implementazioni}

      Una classe pu� implementare una o pi� interfacce ma estende 1 sola classe (di default essa � java.lang.Object)

      Se la classe non � astratta deve fornire un'implementazione per tutti i metodi presenti nelle interfacce che implementa altrimenti essa viene detta astratta.

  \section{I tre principi}
    \subsection{Incapsulamento}
      li oggetti nasconono il loro stato e parte del loro comportamento

    \subsection{Ereditariet�}
      Ogni sottoclasse eredit tutte le propriet� della/delle superclassi

    \subsection{Polimorfismo}
      Stessa interfaccia anche per dtipi di dati diversi.

  \section{Conversioni automatiche di tipo}
    \subsection{Promozioni}
      \begin{itemize}
        \item byte -> short, int long, float, double
        \item short -> int, long, float, double
        \item int -> long, float, double
        \item long -> float, double
        \item float -> double
        \item double -> double
        \item char -> int, long, float, double
      \end{itemize}

    \section{Casting}
      Il casting � necessario quando non c'� la promozione di tipo automatica. Il cast potrebbe anche servire per fare tutto ci� che l'ereditariet�/polimorfismo non ci permette di fare.

      Esempio:
      \begin{lstlisting}
        Animale a = \ldots;
        Gatto mao = \ldots;
        a = mao; //Ok assegnazione polimorfica
        mao = (Gatto)a; // corretto ricastare su una classe successiva, staticamente e sintatticamente corretto, tuttavia potrebbero accadere problemi in dinamica.
      \end{lstlisting}

      Per scrivere un programma pi� robusto dovremmo usare l'operatore \emph{istanceof}

      \begin{lstlisting}
        Object a;
        int i = System.in.read();
        if (i > 0)
          a = new String();
        else
          a = new Integer(5);

        if (a instanceof String) return a.equals(``abcd'');
      \end{lstlisting}

      \subsection{instanceof e equals}
        \begin{lstlisting}
          public boolean equals(Object o) {
            if (!(o instanceof Data)) return false; // risoluzione ricorrente
            Data d = (Data)o;
            return (giorno == d.giorno && mese == d.mese && anno == d.anno); 
          }
        \end{lstlisting}

  \section{Container}
    \subsection{ArrayList}
      gli arraylist sono contenitori ``estendibili'' e ``accorciabili'' dinamicamente. Originariamente potevamo avere un ArrayList (prima vector) che per essere generico era un contentitore di Object. Questo aveva 2 problemi:
      \begin{itemize}
        \item Potevamo creare collezioni che potevano contenere ``tutto''
        \item Avendo degli object per poterli usare si doveva ricastare ogni elemento dell'array.
      \end{itemize}

      In java dalla versione 5 in poi hanno introdotto i Generics. In realt� qualcuno ha definito la classe ArrayList a meno di un parametro.

      \begin{lstlisting}
        import java.util.ArrayList;

        ArrayList<Person> team = new ArrayList<Person>(); // quello che mettiamo nel generics deve essere una classe e non pu� essere un tipo base.

        team.add(new Person(``Bob''));
        team.add(new Person(``Joe''));

        team.size();
      \end{lstlisting}

      L'esempio sopra definisce un contenitore con una relazione d'ordine che mi permetta di gestire nth elementi senza pensare di allocare espandere o preoccuparmi della memoria.

      metodi esposti:
      \begin{itemize}
        \item size() -> mi da la dimensione del contentuto e non la dimensioen del contentitore
        \item add(element) -> aggiunge un elemento all'arraylist, la add � additiva e aggiunge un elemento.
        \item add(ith, element) -> aggiunge un elemento in poszione ith.
        \item get(ith) -> prende l'elemento i-esimo
        \item set(ith, content) -> mi consente di mettere un elemento in una posizione particolare. la set � sostitutiva.
        \item remove(ith) -> rimuove l'elemento iesimo.
      \end{itemize}

      ovviamente posso usare il foreach con l'arraylist.

  \section{Generics}
    \begin{lstlisting}
      List<String> ls = new ArrayList<String>(); // Corretto, poiche ArrayList implementa la List
      List<Object> lo = ls; // ERRORE DI COMPILAZIONE
      lo.add(new Object());+
      String s = ls.get(0); // ERRORE: tenta di assegnare un Object a una string!
    \end{lstlisting}

    Se Gen<T> � una classe generica e ClassB � una sottoclasse di ClassA, allora Gen<ClassB> non � sottoclasse di Gen<ClassA>. In realt� potremmo usare i generici anche per scrivere e non solo per creare dei tipi.

    Supponiamo di voler scrivere un metodo statico che faccia da conversione tra un'array e una collezione.

    \begin{lstlisting}
      static <T> void fromArrayToCollection(T[] a, Collection<T> c) {
        for (T o : a) { c.add(o); }
      }
    \end{lstlisting}

    Esempio:
    \begin{lstlisting}
      String[] sa = new String[100];
      Collection<String> cs = new ArrayList<String>( );
      fromArrayToCollection(sa, cs);
      Collection<Object> co = new ArrayList<Object>( );
      fromArrayToCollection(sa, co;
      Integer[ ] ia = new Integer[100];
      fromArrayToCollection(ia, cs);
    \end{lstlisting}

  \section{IO Formattato}
    \subsection{Input}
      Anche l'input � molto carino ma non � simmetrico rispetto all'output.

      System.in, dobbiamo definire un oggetto di appoggio dalla classe Scanner che ci offre i seguenti metodi:
      \begin{itemize}
        \item nextLine
        \item next
        \item nextInt
        \item nextDouble
        \item hasNext
        \item hasNextInt
        \item hasNextDouble \ldots
      \end{itemize}

      Esempio:
      \begin{lstlisting}
        public static void main(String[ ] args) {
          Scanner in = new Scanner(System.in);
          System.out.println(``Come ti chiami?'');
          String nome = in.nextLine( );
          System.out.println(``Quanti anni hai?'');
          int eta = in.nextInt( );
          System.out.println(``Ciao '' + nome + `` tra un anno avrai '' + (eta + 1) + `` anni'');
      \end{lstlisting}
    \subsection{Output}
      - System.out.println( );
      - System.out.printf( );

  \section{Eccezioni}
    Situazione in cui il programma ha un comportamento eccezionale. Il programma deve poter segnalare l'impossibilit� di procedere nel produrre il risultato al chiamante con un metodo coerente senza utilizzare lo stesso meccanismo di ritorno.

    Soluzioni:
    \begin{itemize}
      \item Terminazione del programma, come molte funzioni C
      \item Restituire un valore, come ad esempio un -1 o valori simili. Tuttavia se usiamo un valore convenzionale da scegliere abbiamo un codominio dei valore di output ridotto.
      \item Variabile globale che dice che c'� stato un errore. Anche questo comporta dei problemi quando c'� pi� di un thread in pi� dovrei controllare quasi ogni istruzione se � successo qualcosa di errato. Gestione centralizzata.
      \item Meccanismo di eccezione come un elemento del linguaggio. Anziche dire che un metodo o termina o si pianta, abbiamo anche la possiblit� che lanci un'eccezione, cio� ci faccia sapere che qualcosa di storto si sia verificato. Vuol dire segnalare al chiamante che qualcosa di strano si � verificato. In questo caso il chiamante ha la possibilit� di gestire come vuole l'eccezione
    \end{itemize}

    In java abbiamo gli oggetti eccezione che se vengono sollevate vengono comunicate al chiamante.

    Esempio:
    \begin{lstlisting}
      try {
        \ldots
      } catch (DivisionByZeroException e) {
          // e � l'oggetto eccezione, qui viene inserito il codice di eccezione
          // Qui siamo in grado di dire di fare qualcosa si verifica quell'eccezione.
      }

      // Qui vengono inserite le istruzioini successive.

    \end{lstlisting}
\end{document}

