\documentclass[a5paper]{article}

% Load packages
\usepackage[latin1]{inputenc}
\usepackage[T1]{fontenc}
\usepackage[italian,english]{babel}

% Set the title
\author{Andrea Brancaleoni}
\title{Lezione3}
\date{2012/03/07 10:32:22}

% Start Document
\begin{document}
  \maketitle

  \section{Argomenti}
    I parametri base passati alle funzioni sono copiati, mentre quelli non base vengono passati per indirizzo.

    Per duplicare un oggetto ho bisogno di copiare a uno a uno i campi semplici. Le String poich� immutabili tuttavia hanno l'efficienza dei ref e la comodit� dei tipi base.

  \section{Attributi}
    \begin{enumeration}
      \item possiamo accedere a tutti gli oggetti della classe se � definito l'attributo statico.
      \item Tutti gli oggetti possono accedere agli attributi statici e tutti possono modificarlo
      \item usato ad esempio quando si vuole mettere un contatore (per esempio di istanza)
      \item Possiamo fare la stessa cosa per i metodi, un metodo � statico quando pu� essere invocato se non c'� bisogno di un oggetto a se associato per essere invocato. Per questo motivo all'interno di esso non � definito l'oggetto $this$.
        \begin{itemize}
          \item potremmo o invocarlo all'interno delle funzioni dei metodi, oppure dovremmo usarlo premettendo il nome della classe, <NomeClasse>.<Nomemetodo>
          \item oppure potremmo chiamare il metodo statico solo col suo nome. Esattamente come una funzione.
        \end{itemize}
        Un metodo static pu� accedere ai soli attributi e metodi (campi) statici della classe
        Altrimenti pu� accedere sia ai campi statici che non.

        P.S. Non usare printf, ha fatto interdere che non gli andrebbe bene.
    \end{enumeration}

  \section{IO}
    La funzione println fa introspection sui tipi dati.

  \section{Enum}
    Dalla versione 6 di Java si possono creare dei tipi enumerati, sono immutabili e non istanziabili

  \section{End}
    Alla prossima puntata.

\end{document}

