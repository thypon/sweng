\documentclass[a5paper]{article}

% Load packages
\usepackage[latin1]{inputenc}
\usepackage[T1]{fontenc}
\usepackage[italian,english]{babel}
\usepackage{listings}
\lstset{language=java}

% Set the title
\author{Andrea Brancaleoni}
\title{Lezione4}
\date{2012/03/12 13:35:38}

% Start Document
\begin{document}
  \maketitle

  \section{Overloading di metodi}
    \begin{itemize}
      \item All'interno della stessa classe possono esservi pi� metodi con lo stesso nome purch� si distinguano per numero e/o tipo di parametri. Il tipo tuttavia non basta a distinguere due metodi.
      \item L'intestazione di un metodo comprende il numero, il tipo e la posizione dei parametri; non include il tipo del valore restituito.
      \item Utile per definire funzioni con codice differente ma con effetti simili su tipi diversi.
    \end{itemize}

    Noi sovraccarichiamo il nome, mentre in C il nome era identificatore di un comportamento. Guardando solo il nome non siamo in grado di definire il comportamento ma dobbiamo anche conoscere i suoi parametri.

    Molto interessante per un linguaggio ad oggetti per overload del costruttore ad esempio.

    \begin{lstlisting}
      public class Prova {
          public int max(int a, int b, int c) { . . . }
          public double max(double a, double b) { . . . }
          public int max(int a, int b) { . . . }
      }
    \end{lstlisting}

    Ogni volta viene chiamata la funzione giusta. Tuttavia non possiamo avere

    \begin{lstlisting}
      public static void main(String[] args) {
        Prova p = new Prova();

        p.max(2, 3, 5);
        p.max(2.3, 3.14);
        p.max(2, 3);
      }
    \end{lstlisting}

    \section{Costruzione e Distruzione}
      \begin{itemize}
        \item I costruttori spesso utilizzano l'overload
        \item Svolge 2 ruoli:
          \begin{itemize}
            \item Alloca la memoria
            \item Da dei valori vagamente significativi
          \end{itemize}
        \item A differenza del C, in Java non � necessario deallocare esplicitamente gli oggetti
      \end{itemize}

      Esempio:

      \begin{lstlisting}
        public Data(int g, int m, int a) {
          giorno = g;
          mese = m;
          anno = a;
        }
      \end{lstlisting}

      Il costruttore � un metodo come tutti gli altri che ha come parametri i parametri dati, e all'interno del corpo dovremmo andare a inizializzare i valori degli attributi di quel progetto. Nel momento in cui andiamo ad invocare il costruttore l'allocazione della memoria avverr� in automatcio, dentro al costruttore andiamo a definire come gli attributi devono essere gestiti.

      In Java possiamo avere tutti i costruttori che ci interessano. Con la regola che il costruttore si deve chiamare con la stesso nome della classe. Tutto questo lo si pu� fare come lo riteniamo pi� opportuno. L'unica convezione di java  la seguente: se non dichiariamo costruttori il costruttore di default sar� invocato.

      Esempio:

      \begin{lstlisting}
        public class C {
          private int i1, i2;
        }

        public static void main(String[] args) {
          C x = new C();
          x = new C(2,3); // Sbagliato
        }
      \end{lstlisting}

      Tuttavia appena definisco un costruttore perdo il costruttroe di default.

      \begin{lstlisting}
        public class Automobile {
          private String colore, marca, modello;
          private int cilindrata, numPorte;
          private boolean acceso;

          public Automobile(String col, String mar, String mod) {
            colore = col;
            marca = mar;
            modell = mod;
          }

          public Automobile() {
            colore = marca = modello = null;
            cilindrata = numPorte = 0;
          }
          . . .
        }
      \end{lstlisting}

      \begin{itemize}
        \item � possibile invocare un costruttore dall'interno di un altro tramite la notazione this.metodo.
        \item Tuttavia this deve essere la prima istruzione.
      \end{itemize}

      \begin{lstlisting}
        import java.util.Calendar;

        . . .

        public Data(int g , int m, int a) {
          if (dataValida(g, m, a)) { // dataValida � un metodo statico.
            giorno = g;
            mese = m;
            anno = a;
          } else . . .
        }

        . . .

        public Data(int g, int m) { // Prende giorno e mese che vogliamo e l'anno corrente
          this(g, m, Calendar.getInstance().get(Calendar.YEAR)); // Il terzo parametro restituisce l'anno corrente.
        }
      \end{lstlisting}


    \section{Operatore di confronto}
      L'operatore di confronto == confronta i valori di riferimenti e non degli oggetti. Se noi abbiamo delle variabili non referenza andiamo a confrontare le variabili in caso contrario andiamo a confrontare i riferimenti.

      \begin{lstlisting}
        Data d1 = new Data(1, 12, 2001);
        Data d2 = new Data(1, 12, 2001);

        if (d1 == d2) { // false poiche i due oggetti sono diversi, test di identit�
          . . .
        }

        // La stessa cosa quando assegiamo i due oggetti uno sull'altro.

        d1 = d2;

        // Con gli oggetti generalmente si preferisce chiamare il metodo .equals.

      \end{lstlisting}

      \subsection{Confronto di uguaglianza}
        \begin{itemize}
          \item Metodo equals consente di confrontare se due oggetti sono uguali (nel senso che hanno lo stesso valore). E' un test di uguaglianza mentre == � un test di identit�. Tuttavia non sempre esiste una relazione di equivalenza ovvia tra classi autodefinite. Metaforicamente, Fa un confronto ancora tra valori e riferimenti uno a uno. Nella pratica controlla prima l'identit� per rapidit� e poi l'uguaglianza.
          \item Dice se due oggetti sono equivalenti.
        \end{itemize}

        \begin{lstlisting}
          String s1 = ``Luciano'';
          String s2 = ``Giovanni'';

          s1.equals(s2); // False
          String a = new String(``Ciao'');
          String b = new String(``Ciao'');
          a.equals(b);
        \end{lstlisting}

  \section{Tipi riferimento e tipi primitivi}
    \begin{itemize}
      \item i tipi primitivi sono comodi, ma a volte � comodo usarli come riferimento per omogeneit�
      \item Java fornisce delle classi predefinite.
        \begin{itemize}
          \item Integer, Character, Float, Long, Short, Double
          \item Un oggetto Integer contiene un int, ma viene inizializzato solo con i costruttori, la classe Integer � immutabile, esattamente come la String e tutti le altri classi duali a i tipi base.
        \end{itemize}
    \end{itemize}

    Esempio:
    \begin{lstlisting}
      Integer i; // Il default � null
      i = new Integer(5); // i � un riferimento a un oggetto che contiene 5
      Integer x = i; // Due riferimenti che puntano allo stesso oggetto il cui contenuto � 5
      int y = x.intValue(); // Vecchio metodo
      i = y; // boxing automatico
      y = i; // unboxing automatico
      i = 3; // stessa cosa di sopra.
    \end{lstlisting}

  \section{Catene puntate}
    un metodo pu� avere tre risultati:
    \begin{itemize}
      \item O non restituisce nulla (void), magari side effect
      \item oppure pu� restituire un valore
      \item ritorna this; in questo caso � possibile la catena di invocazione.
    \end{itemize}

    Esempio:
    \begin{lstlisting}
      System.out.println(); 
      // out � un attributo pubblico 
      // (statico) di classe System
      // La classe di out fornisce il metodo println();

      String b = new String(``Ciao'');
      String a = b
          .substring(1)
          .substring(2);
      System.out.pringln(a);

      // L'operatore . � associativo a sinistra
    \end{lstlisting}

    \begin{lstlisting}
      public class Persona {
        private String name;
        private Persona padre;

        private int et�;

        public Persona(String n) {
          name = n; padre = null; et� = 0;
        }

        public void setPadre(Persona p) {
          padre = p;
        }

        public void setAta(int e) { eta = e; }
      }

      public class Esempio {
        public static void main(String[] args) {
          Persona paolo, pietro;
          int i = 20;
          paolo = new Persona(``Paolo'');
          paolo.setEta(i);
          i = 30;
          pietro = new Persona(``Pietro'');
          pietro.setEta(i);
          paolo.setPadre(pietro);
          pietro.setEta(i + 1);
        }
      }
    \end{lstlisting}

    \section{Ereditariet�}
      Il professore insiste a chiamarla ereditariet� e non eredit�, immagino non voglia vedere la parola eredit� nel compito.
      \begin{itemize}
        \item � possibile stabilire una relazione di ``sottoclassi di'' fra le classi di un programma Java
          \begin{itemize}
            \item Relazione d'ordine parziale (riflessiva e transitiva)
            \item public class B extends A { \ldots } // Dal punto di vista sintattico.
          \end{itemize}
        \item A classe base o antenato o padre o superclasse
        \item B � classe figlia discendente o semplicemente sottoclasse
      \end{itemize}

      Quando andremo a definire B se la deriveremmo da A andremmo solo a definire le differenze dalla classe A.

      \subsection{Relazione di ereditariet�}
        \begin{itemize}
          \item La sottoclasse eredita tutta l'implementazione (attributi e metodi) della superclasse. Gli attributi e metodi della superclasse sono implicitamente definiti anche nella sottoclasse. Se andassimo per errore a definire nella sottoclasse un attributo con lo stesso nome della sottoclasse ci troveremmo con due attributi con lo stesso nome.
          \item Una sottoclasse pu� aggiungere nuovi attributi, nuovi metodi ma pu� anche ridefinire i metodi che eredita dalla superclasse, ovviamente i metodi devono invariare numero e tipo dei parametri (ci� � detto overriding). Se ci� non accade questo � un semplice overloading.
        \end{itemize}

        Esempio:
        \begin{lstlisting}
          public class Automobile {
            private String modello;
            private boolean accesa;
            public Automobile(String modello) {
              this.modello = modello;
              this.accesa = false;
            }

            public void accendi() {accesa = true;}
            public boolean puoPartire() { return accesa; }
          }

          public class AutomobileElettrica extends Automobile {
            public void ricarica() {}
          }
        \end{lstlisting}

        Gerarchia a pi� livelli, in java il figlio pu� avere al pi� un ancestors bench� non possa avere pi� di un singolo padre.
        // Disegno1


\end{document}

